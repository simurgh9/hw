% Created 2021-01-08 Fri 20:42
% Intended LaTeX compiler: xelatex
\documentclass{homework}
        

\class{CS 3141: Prof. Kamil's Algorithm Analysis}
\address{Bayt El-Hikmah}
\author{Musa Al`Khwarizmi}
\date{\today}
\title{Minimal Complete Document}
\hypersetup{
 pdfauthor={Musa Al`Khwarizmi},
 pdftitle={Minimal Complete Document},
 pdfkeywords={},
 pdfsubject={},
 pdfcreator={Emacs 26.3 (Org mode 9.4)}, 
 pdflang={English}}
\begin{document}

\maketitle

\question Write down sets in order of containment.
\label{sec:org3bcb8f3}

We pretend that equivalence classes are just numbers.
\[
  \C \supset \R \supset \Q \supset \Z \supset \N \supset \P \not\supset (\GF[7] = \modulo[7]) \supset \{\nil\}
\]

\question Give an example element of \(\O(n)\).
\label{sec:org61c5625}

Take \(11n \in \O(n)\).

\question Find roots of \(x^2- 8x = 9\).
\label{sec:orgd5cb924}

We proceed by factoring,

\begin{align*}
  x^2- 8x - 9         & = 9-9         &  & \text{Subtract 9 on both sides.}         \\
  x^2- x + 9x - 9     & = 0           &  & \text{Breaking the middle term.}         \\
  x(x - 1) + 9(x - 1) & = 0           &  & \text{Pulling out common factors.}       \\
  (x - 1)(x + 9)      & = 0           &  & \text{Pulling out common } (x - 1).      \\
  x                   & \in \{1, -9\} &  & f(x)g(x) = 0 \Ra f(x) = 0 \vee g(x) = 0. \\
\end{align*}

\question Show P \(\?\) NP.
\label{sec:org280f50e}

Let P be zero\ldots{} Sorry.
\end{document}