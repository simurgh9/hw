%% hw.tex Copyright (C) 2020  Ahmad Tashfeen
%% This program comes with ABSOLUTELY NO WARRANTY.
%% This is free software, and you are welcome to redistribute it
%% under certain conditions; read file COPYING for more details.

\documentclass{homework}

\author{Musa Al`Khwarizmi}
\class{CS 3141: Prof. Kamil's Algorithm Analysis}
\date{\today}
\title{\LaTeXe{} Homework Class Documentation \& Testing}
\address{Bayt El-Hikmah}

\graphicspath{{../../media/}} % tex.stackexchange.com/q/139401/215221
\newcommand{\bk}{\textbackslash}
\renewcommand\tt[1]{\texttt{#1}}
\newcommand{\dummy}[1]{\texttt{\bk #1\{\bk frac\{x\}\{2\}\}}}
\newcommand{\amslink}{%
  \href{https://ctan.org/pkg/amsart}{%
    \textit{American Mathematical Society (AMS) Journal Article} %
  }}

% \fold% uncomment this to hide answers.
% \renewcommand\solcolor{black}% uncomment this if you're colour blind.

\begin{document} \maketitle

\question What is this document?

\begin{sol}
  This is a demonstration of my homework class aiming to ease a few of
  the \textit{hassles} of \LaTeX{}. It is an extension of the \amslink
  (\tt{amsart}) class and should have all of its functionality.
\end{sol}

\question What preamble commands does the class have?

\begin{sol}
  We have all that AMS article has, the preamble for this document had:
  \begin{lstlisting}[language={[LaTeX]TeX}]
  ...
  \author{Musa Al`Khwarizmi}
  \class{CS 3141: Prof. Kamil's Algorithm Analysis}
  \date{\today}
  \title{Homework Class Test}
  \address{Bayt El-Hikmah}
  ...
\end{lstlisting}
\end{sol}

\question What symbol shortcuts does it have?

\begin{sol}
  It has the symbols shown in table \ref{symbols},
  \tbl<symbols>{Symbols table.}{
    Macro                  & Symbol       & Macro                           & Symbol                \\
    \tt{\bk C}             & $\C$         & \dummy{floor}                   & $\floor{\frac{x}{2}}$ \\
    \tt{\bk R}             & $\R$         & \dummy{ceil}                    & $\ceil{\frac{x}{2}}$  \\
    \tt{\bk Q}             & $\Q$         & \dummy{near}                    & $\near{\frac{x}{2}}$  \\
    \tt{\bk Z}             & $\Z$         & \dummy{arr}                     & $\arr{\frac{x}{2}}$   \\
    \tt{\bk N}             & $\N$         & \dummy{paren}                   & $\paren{\frac{x}{2}}$ \\
    \tt{\bk P}             & $\P$         & \dummy{brk}                     & $\brk{\frac{x}{2}}$   \\
    \tt{\bk F}             & $\F$         & \dummy{curl}                    & $\curl{\frac{x}{2}}$  \\
    \tt{\bk GF, \bk GF[7]} & $\GF,\GF[7]$ & \dummy{abs}                     & $\abs{\frac{x}{2}}$   \\
    \tt{\bk 0}             & $\nil$       & \tt{\bk modulo[7]}              & $\modulo[7]$          \\
    \tt{\bk O(n)}          & $\O(n)$      & \tt{\bk vec\{v\}}               & $\vec{v}$             \\
    \tt{\bk ?}             & $\?$         & \tt{\bk bijective}              & $\bijective$          \\
    \tt{\bk is}            & $\is$        & \tt{\bk surjective}             & $\surjective$         \\
    \tt{\bk al}            & $\al$        & \tt{\bk injective}              & $\injective$          \\
    \tt{\bk ep}            & $\ep$        & \tt{\bk Ra}                     & $\Ra$                 \\
    \tt{\bk phi}           & $\phi$       & \tt{\bk ra}                     & $\ra$                 \\
    \tt{\bk p}             & $\p$         & \tt{\bk derivative[g]\{f\}}     & $\derivative[g]{f}$   \\
    \tt{\bk D}             & $\D$         & \tt{\bk derivative\{\bk zeta\}} & $\derivative{\zeta}$  \\
  }

  The commands that have twin delimiters expand according to their input,
  \[
    \floor{x}, \ceil{y}, \near{z}, \arr{x,y,z},
    \floor{\frac{x}{2}} < \frac{x}{2} < \ceil{\frac{x}{2}},
    \near{\frac{x}{2}},
    \arr{\frac{x}{2}, \frac{x}{3}, \frac{x}{4}},
    \paren{\frac{x}{2}, \frac{x}{3}, \frac{x}{4}},
    \brk{\frac{x}{2}, \frac{x}{3}, \frac{x}{4}},
    \curl{\frac{x}{2}, \frac{x}{3}, \frac{x}{4}},
    \abs{\frac{x}{2}}
  \]
\end{sol}

\question Are pictures still a pain though?

\begin{sol}
  No! we have,
  \begin{center}
    \tt{\bk img<label>[width]\{caption\}\{path/f0, path/f1, ... , path/fn\}}
  \end{center}
  Or, you can set the path to all the images once with \tt{\bk
    graphicspath\{\{path\}\}} in the preamble and then you can do,
  \begin{center}
    \tt{\bk img<label>[width]\{caption\}\{f0, f1, ... , fn\}}.
  \end{center}
  We did \tt{\bk graphicspath\{\{../media/\}\}} with,
  \begin{center}
    \tt{\bk img<trio>[0.2]\{Al`Khwarizmi\}\{khwarizmi, kitab, page\}}
  \end{center}
  to get the figure \ref{trio}.
  \img<trio>[0.2]{Al`Khwarizmi}{khwarizmi, kitab, page}
\end{sol}

\question What about tables?

\begin{sol}
  The following code to the left gets you the table \ref{smptb}.

  \medskip
  \begin{minipage}{0.45\textwidth}
    \begin{verbatim}
\tbl<smptb>{Sample table.} {
  Linear & Polynomial & Exponential \\
  $x$    & $x^2$      & $2^x$       \\
  1      & 1          &     2       \\
  2      & 4          &     4       \\
  3      & 9          &     8       \\
  4      & 16         &    16       \\
  5      & 25         &    32       \\
  6      & 36         &    64       \\
  7      & 49         &   127       \\
}
    \end{verbatim}
  \end{minipage}
  \hfill
  \begin{minipage}{0.45\textwidth} \hspace{1.9em}
    \begin{tabular}{||c | c | c ||} \hline
      Linear & Polynomial & Exponential \\ \hline
      $x$    & $x^2$      & $2^x$       \\ \hline
      1      & 1          & 2           \\ \hline
      2      & 4          & 4           \\ \hline
      3      & 9          & 8           \\ \hline
      4      & 16         & 16          \\ \hline
      5      & 25         & 32          \\ \hline
      6      & 36         & 64          \\ \hline
      7      & 49         & 127         \\ \hline
    \end{tabular}
    \captionsetup{type=table}
    \captionof{table}{Sample table.}\label{smptb}
  \end{minipage}
\end{sol}

\question Do we have all AMS article environments?

\begin{sol}
  Yes! E. g., the proof environment. Note the fancy \textit{Q.E.D} symbol.

  \vspace*{1em}
  \begin{minipage}{0.45\textwidth}
    \begin{verbatim}
  \begin{proof}
    Four is the sum of two integers.

    $1,3 \in \Z$ and $1+3=4$.
  \end{proof}
    \end{verbatim}
  \end{minipage}
  \hfill
  \begin{minipage}{0.45\textwidth} \hspace{1.9em}
    \begin{proof}
      Four is the sum of two integers.

      $1,3 \in \Z$ and $1+3=4$.
    \end{proof}
  \end{minipage}
\end{sol}

\question\label{cardinality} To show citations and references to
custom labels: What is the cardinality of $\N$?

\begin{sol}
  It is $\aleph_0$ \cite{arlinghaus1996part} (\tt{\bk
    cite\{arlinghaus1996part\}}). See also question \ref{custom-index}
  (\tt{\bk ref\{custom-index\}}).
\end{sol}

\question What headlines does this class have?

\begin{sol}
  We have the following hierarchy of headlines:
  \begin{enumerate}
    \item \tt{\bk question[custom-ind]}
    \item \tt{\bk section\{name\}}
    \item \tt{\bk subsection\{name\}}
    \item \tt{\bk subsubsection\{name\}}.
  \end{enumerate}

  Since we inherited all the section commands from AMS article, we can
  also use their stared variants. We demonstrate these bellow,
  \section{Section}
  \subsection{Subsection}
  \subsection{Subsection}
  \section{Another Section}
  \subsection{Subsection}
  \subsubsection*{Subsubsection} This is a started section.
  \subsubsection{Subsubsection} We end here.

\end{sol}

\question Are all headlines preceded by the question number they are under?

\begin{sol}
  Yes, they are preceded by the index of question they are under.

  \section{Section}
  \subsection{Subsection}
  \section{Another Section}
  \subsection{Subsection}
  \subsubsection{Subsubsection}
  \subsubsection{Subsubsection} We end here.
\end{sol}

\question[IX]\label{custom-index} Is the cardinality of Naturals and
Reals the same because they are both infinite?

\begin{sol}
  No, the cardinality of $\R$ is greater because they are also
  uncountable. See also question \ref{cardinality}.
\end{sol}

% \vspace{8\baselineskip}% uncomment to see the next question move to the next page
\question How much space does a question need at the end of a page?

\begin{sol}
  Starting from the question statement, we need at least 8 lines left
  on the page or the question moves to next page.

    [4]

    [5]

    [6]

    [7]

    [8]
\end{sol}

\question What is a complete minimal example?

\begin{sol}
  In listing \ref{doc} we show a complete document using \tt{homework.cls},

  \lstinputlisting[
    language={[LaTeX]TeX},
    caption={Complete \LaTeX{} Document},
    label=doc]
  {minimal.tex}

  This will get you a document looking like the one in figure
  \ref{min}. Note that it appears bigger for illustration purposes
  only.

  \img<min>[0.8]{Out document from listing \ref{doc}.}{./minimal}
\end{sol}

% Tashfeen should run C-c C-s LuaLatex BibTex LuaLatex LuaLatex
% \cleardoublepage
\phantomsection
\addcontentsline{toc}{part}{References}
\bibliographystyle{plain}
\bibliography{citations}

% \tableofcontents
\end{document}
